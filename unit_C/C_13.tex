\Exercise[number={13}]
We intend to use a scalar measure \(x\) (scalar) to decide whether a
given individual is affected (\(w_1\)) or not (\(w_2\)) by a certain
pathology. Assume that the two classes have the same prior probability.
We know that \(x\) can only have three values: \(x\in\{0,1,2\}\) and that
for the two classes we respectively have:
\begin{align*}
    \begin{matrix}
        p(x=1|w_1)=q_{11} \\ p(x=1|w_2)=q_{12}
    \end{matrix}
    \text{ and }
    \begin{matrix}
        p(x=2|w_1)=q_{21} \\ p(x=2|w_2)=q_{22}
    \end{matrix}
    \quad
    \bigl[
        \text{How to compute }
        p(x=0|w_1)
        \text{ and }
        p(x=0|w_2)
        \text{?}
    \bigr]
\end{align*}
Design a Bayes classifier, which assigns the correct class to each
observed \(x\). Depending on the measured value of \(x\): 0 or 1 or 2,
specify (in terms of \(q_{11}\), \(q_{12}\), \(q_{21}\), \(q_{22}\)) the
relevant decision criterion.

\Answer[number={13}]
First of all, let's easily derive the following probabilities:
\begin{align*}
    p(x=0|w_1)=1-(q_{11}+q_{21})
    \quad\text{and}\quad
    p(x=0|w_2)=1-(q_{12}+q_{22})
\end{align*}
Now, the multi-class Bernoulli distribution is to be written, but first
it is necessary to compute the exponents. Let's denote
\(p_2(x)=ax^2+bx+c\) and then:
\begin{align*}
    \begin{cases}
        p_2(0)=0\\p_2(1)=1\\p_2(2)=0
    \end{cases}
    \Rightarrow
    \begin{cases}
        c=0\\a+b+c=1\\4a+2b+c=0
    \end{cases}
    \Rightarrow
    \begin{cases}
        c=0\\a=1-b\\4-4b+2b=0
    \end{cases}
    \Rightarrow
    \begin{cases}
        c=0\\a=-1\\b=2
    \end{cases}
\end{align*}
Therefore, the first exponent is given by \(p_2(x)=-x^2+2x=x(2-x)\).
\begin{align*}
    \begin{cases}
        p_2(0)=0\\p_2(1)=0\\p_2(2)=1
    \end{cases}
    \Rightarrow
    \begin{cases}
        c=0\\a+b+c=0\\4a+2b+c=1
    \end{cases}
    \Rightarrow
    \begin{cases}
        c=0\\a=-b\\-4b+2b=1
    \end{cases}
    \Rightarrow
    \begin{cases}
        c=0\\a=\frac{1}{2}\\b=-\frac{1}{2}
    \end{cases}
\end{align*}
The second exponent is \(p_2(x)=\frac{1}{2}x^2-\frac{1}{2}x=\frac{x}{2}(x-1)\).
\begin{align*}
    \begin{cases}
        p_2(0)=1\\p_2(1)=0\\p_2(2)=0
    \end{cases}
    \Rightarrow
    \begin{cases}
        c=1\\a+b+c=0\\4a+2b+c=0
    \end{cases}
    \Rightarrow
    \begin{cases}
        c=1\\a=-1-b\\-4-4b+2b+1=0
    \end{cases}
    \Rightarrow
    \begin{cases}
        c=1\\a=\frac{1}{2}\\b=-\frac{3}{2}
    \end{cases}
\end{align*}
Lastly, the third exponent is \(p_2(x)=\frac{1}{2}x^2-\frac{3}{2}x+1=\frac{1}{2}(x-1)(x-2)\).\\
Then, the multi-class Bernoulli distributions can be written as:
\begin{align*}
    p(x|w_1)
    &=q_{11}^{x(2-x)}\cdot q_{21}^{\frac{x}{2}(x-1)}\cdot\bigl[1-(q_{11}+q_{21})\bigr]^{\frac{1}{2}(1-x)(2-x)}\\
    p(x|w_2)
    &=q_{12}^{x(2-x)}\cdot q_{22}^{\frac{x}{2}(x-1)}\cdot\bigl[1-(q_{12}+q_{22})\bigr]^{\frac{1}{2}(1-x)(2-x)}
\end{align*}
Now it is easy to write down the decision criterion:
\begin{align*}
    z(x)
    &=\log{\frac{p(x|w_1)}{p(x|w_2)}}+\cancel{\log{\frac{Pr(w_1)}{Pr(w_2)}}}
    \log{p(x|w_1)}-\log{p(x|w_2)}\\
    &=x(2-x)\log{q_{11}}+\frac{x}{2}(x-1)\log{q_{21}}+\frac{1}{2}(1-x)(2-x)\log{\bigl[1-(q_{11}+q_{21})\bigr]}+\\
    &\quad-x(2-x)\log{q_{12}}-\frac{x}{2}(x-1)\log{q_{22}}-\frac{1}{2}(1-x)(2-x)\log{\bigl[1-(q_{12}+q_{22})\bigr]}\\
    &=x(2-x)\log{\frac{q_{11}}{q_{12}}}+\frac{x}{2}(x-1)\log{\frac{q_{21}}{q_{22}}}+\frac{1}{2}(1-x)(2-x)\log{\frac{1-(q_{11}+q_{21})}{1-(q_{12}+q_{22})}}
\end{align*}
Notice that if \(w_1\) is to be chosen, then \(z(x)>0\). Three cases
are possible:
\begin{enumerate}
    \item\quad \(x=1\Rightarrow z(x)=\log{\frac{q_{11}}{q_{12}}}>0\Longleftrightarrow q_{11}>q_{12}\)
    \item\quad \(x=2\Rightarrow z(x)=\log{\frac{q_{21}}{q_{22}}}>0\Longleftrightarrow q_{21}>q_{22}\)
    \item\quad \(x=0\Rightarrow z(x)=\log{\frac{1-(q_{11}+q_{21})}{1-(q_{12}+q_{22})}}>0\Longleftrightarrow \cancel{1}-(q_{11}+q_{21})>\cancel{1}-(q_{12}+q_{22})\Rightarrow q_{11}+q_{21}<q_{12}+q_{22}\)
\end{enumerate}
On the opposite, the \(w_2\) class is selected when \(z(x)<0\), thus
all the inequalities are inverted.