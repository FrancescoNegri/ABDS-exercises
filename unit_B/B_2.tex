\Exercise[number={2}]
A set of data (in \(\mathbf{R}^2\)) is described by a probability density
function of Gaussian type \(x \sim \mathcal{N}(\mu, \Sigma)\). Prove that
the contour lines of \(p(x)\), that is the locus of the points \(x\) for
which \(p(x) = constant\), describe an ellipse centered in and whose
half-axes are specified by eigenvectors and eigenvalues of \(\Sigma\).

\Answer[number={2}]
TBD