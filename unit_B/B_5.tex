\Exercise[number={5}]
From his cell, a prisoner observes through a window \(N\) stars, in positions
\(D = \{(x_i, y_i), i = 1...N\}\). He cannot see the edges of the window
because the room is totally dark apart from the stars. 
Assuming that the window is square and the positions of the stars are
independent and uniformly distributed in \(XY\) space, what can be said about
the position \((x_0, y_0)\) and the length \(L\) of the side of 
the window? [Hint: This is a probability density estimation problem.
Just write an expression for likelihood ...]

\Answer[number={5}]
First of all, it might be useful to sketch both the problem and the
probability density function:
\begin{figure}[H]
    \includegraphics[scale=0.65]{B_5}
    \centering
\end{figure}
The goal of the exercise consists in finding the smallest window (thus
smallest \(L\)) such that all the \(N\) stars are visible from it.
Notice that integral of the probability density is supposed to be 1,
hence the volume of the box indicated in the right-side plot should be 1,
leading to the following:
\[
    V=1 \quad\text{and}\quad V=Ah=L^2h \Rightarrow L^2h=1 \Rightarrow h=\frac{1}{L^2}
\]
The a-posteriori probability is to be written:
\[
    p((x_0,y_0),L|D)=\frac{p(D|(x_0,y_0),L)p((x_0,y_0),L)}{p(D)}
    \propto
    p(D|(x_0,y_0),L) = L((x_0,y_0),L|D)
\]
The likelihood is evaluated as:
\[
    L((x_0,y_0),L|D) = \prod_{i=1}^{N}p((x_i,y_i)|(x_0,y_0),L)=\prod_{i=1}^{N}
    \begin{cases}
        \begin{matrix}
            h\,\Bigl(=\frac{1}{L^2}\Bigr) && (x_i, y_i)\in \text{window} \\
            0 && (x_i,y_i)\notin \text{window}
        \end{matrix}
    \end{cases}
\]
By assuming that all the \(N\) stars are seen by the window (otherwise the
likelihood product would be 0 even if just one star is not seen), it can
be said that:
\[
    L((x_0,y_0),L|D) = \biggl(\frac{1}{L^2}\biggr)^N
\]
Therefore, the likelihood is maximized when \(L\to{0}\) (as small as
possible), but still able to allow to see all the stars.