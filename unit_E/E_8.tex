\Exercise[number={8}]
Consider the following linear dynamical system:
\begin{align*}
    \begin{cases}
        x_{i+1}=A\cdot x_i + w_i \\ y_i=C\cdot x_i + v_i
    \end{cases}
\end{align*}
where \(p(w_i)=\mathcal{N}(w_i|0,Q)\) and \(p(v_i)=\mathcal{N}(v_i|0,R)\).\\
Let \(A=I_2\) (identity 2x2 matrix), \(C=[1\quad1]\), \(R=\sigma_1^2\),
\(Q=\sigma^2I_2\).\\
Also let (initial conditions): \(P_0^+=I_2\) and \(x_0^+=0\).
\begin{itemize}
    \item\quad a. Calculate \(K_1\) (Kalman gain at time \(i=1\)).
    \item\quad b. Calculate \(x_1^+\) (posterior state estimate at
    time \(i=1\)).
\end{itemize}

\Answer[number={8}]
a. Let's recall the main formulas for the Kalman Filter:
\begin{align*}
    \begin{matrix}
        {} && K_t=\hat{P}_t^-C^T\bigl(C\hat{P}_t^-C^T+R\bigr)^{-1} && {}\\
        \hat{x}_t^-=A\hat{x}_{t-1}^+ && {} && \hat{x}_t^+=\hat{x}_t^-+K_t(y_t-C\hat{x}_t^-)\\
        \hat{P}_t^-=A\hat{P}_{t-1}^+A^T+Q && {} && \hat{P}_t^+=\hat{P}_t^--CK_t\hat{P}_t^-
    \end{matrix}
\end{align*}
Now the variance of the prior error can be computed:
\begin{align*}
    \hat{P}_1^-
    =A\hat{P}_{0}^+A^T+Q
    =I_2\cdot I_2\cdot I_2 + \sigma^2I_2
    =I_2+\sigma^2I_2
    =
    \begin{bmatrix}
        \sigma^2+1&&0\\0&&\sigma^2+1
    \end{bmatrix}
\end{align*}
Then, the Kalman gain is:
\begin{align*}
    K_1
    =\hat{P}_1^-C^T\bigl(C\hat{P}_1^-C^T+R\bigr)^{-1}
    &=
    \begin{bmatrix}
        \sigma^2+1&&0\\0&&\sigma^2+1
    \end{bmatrix}
    \begin{bmatrix}
        1\\1
    \end{bmatrix}
    \Biggl(
        \begin{bmatrix}
            1&&1
        \end{bmatrix}
        \begin{bmatrix}
            \sigma^2+1&&0\\0&&\sigma^2+1
        \end{bmatrix}
        \begin{bmatrix}
            1\\1
        \end{bmatrix}
        +\sigma_1^2
    \Biggr)^{-1}\\
    &=
    \begin{bmatrix}
        \sigma^2+1\\\sigma^2+1
    \end{bmatrix}
    \Biggl(
        \begin{bmatrix}
            1&&1
        \end{bmatrix}
        \begin{bmatrix}
            \sigma^2+1\\\sigma^2+1
        \end{bmatrix}
        +\sigma_1^2
    \Biggr)^{-1}\\
    &=
    \begin{bmatrix}
        \sigma^2+1\\\sigma^2+1
    \end{bmatrix}
    \bigl(2\sigma^2+2+\sigma_1^2\bigr)^{-1}
    =
    \frac{1}{2\sigma^2+2+\sigma_1^2}
    \begin{bmatrix}
        \sigma^2+1\\\sigma^2+1
    \end{bmatrix}
\end{align*}
b. The prior state estimate is to be computed as follow:
\begin{align*}
    \hat{x}_1^-
    =A\hat{x}_0^+
    =
    \begin{bmatrix}
        1&&1
    \end{bmatrix}
    \begin{bmatrix}
        0\\0
    \end{bmatrix}
    =
    \begin{bmatrix}
        0\\0
    \end{bmatrix}
\end{align*}
Therefore, the posterior state estimate at \(t=1\) is expressed as:
\begin{align*}
    \hat{x}_1^+
    =\cancel{\hat{x}_1^-}+K_1(y_1-\cancel{C\hat{x}_1^-})
    =K_1y_1
    =
    \frac{y_1}{2\sigma^2+2+\sigma_1^2}
    \begin{bmatrix}
        \sigma^2+1\\\sigma^2+1
    \end{bmatrix}
\end{align*}