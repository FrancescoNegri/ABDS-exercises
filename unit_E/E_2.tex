\Exercise[number={2}]
A virus exists in \(N\) different variants. At each generation the virus
mutates with probability \(0<p<1\) into a different variant, chosen at
random among the remaining ones. Calculate the probability that after \(n\) 
generations the virus type has not changed with respect to the initial
type (generation 1). 

\Answer[number={2}]
If \(p\) is the probability for a mutation to occur between two subsequent
generation, it is easy to see that the probability for a mutation not to
occur is just \(1-p\).\\
Said so, in order not to have a mutation between two generations:
\begin{align*}
    p(S_t|S_{t-1})=1-p
\end{align*}
By extending the formula to \(n\) generations one may simply obtain:
\begin{align*}
    p(S_n|S_{1:n-1})=\prod_{i=1}^{n}(1-p)=(1-p)^n
\end{align*}